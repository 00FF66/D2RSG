\subsubsection{Рынок ресурсов}
\label{resourcemarket}
Описывает рынок, его обменный курс, запасы ресурсов и потенциальную охрану.

Обязательных полей нет, пустая таблица описывает рынок с обменным курсом по-умолчанию, без ресурсов и охраны.

Необязательные поля:
\begin{itemize}
\item \texttt{exchangeRates} - строка, содержащая скрипт обменного курса, специфичного для рынка.
\item \texttt{stock} - список \hyperref[marketresource]{\selectlanguage{Russian}ресурсов рынка}.
\item \texttt{guard} - отряд охраняющий вход на рынок. См. \hyperref[group]{\selectlanguage{Russian}Группа / Отряд}
\item \texttt{aiPriority} - приоритет рынка для ИИ в диапазоне \texttt{[0 : 6]}
\item \texttt{name} - название рынка. По умолчанию будет использовано название из \texttt{Marketname.dbf}
\item \texttt{description} - описание рынка. По умолчанию будет использовано описание из \texttt{Marketname.dbf}
\end{itemize}

Пример:\\
Рынок с бесконечным запасом золота, 100 единицами маны жизни, 200 - 300 единиц маны преисподней и 500 - 550 единиц маны смерти

\begin{figure}[H]
\lstinputlisting{docExamples/marketExample1.lua}
\end{figure}

Рынок с бесконечным запасом маны и специальным обменным курсом который позволяет игрокам менять золото на любой вид маны в соотношении 10 к 1

\begin{figure}[H]
\lstinputlisting{docExamples/marketExample2.lua}
\end{figure}