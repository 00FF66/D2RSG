\subsubsection{Торговец}
\label{merchant}
Описывает лавку торговца, его товары и потенциальную охрану.

Обязательные поля:
\begin{itemize}
\item \texttt{goods} - ассортимент предметов торговца. См. \hyperref[loot]{\selectlanguage{Russian}Награда}. Торговцы в игре не продают драгоценности (\texttt{Item.Valuable}), поэтому генератор \textbf{\selectlanguage{Russian}никогда} не создаст предметы этого типа среди случайных. Если такие предметы по какой-то причине все же необходимы в ассортименте торговца, следует указать их в списке обязательных предметов
\end{itemize}

Необязательные поля:
\begin{itemize}
\item \texttt{guard} - отряд охраняющий вход в лавку торговца. См. \hyperref[group]{\selectlanguage{Russian}Группа / Отряд}
\item \texttt{aiPriority} - приоритет торговца для ИИ в диапазоне \texttt{[0 : 6]}
\item \texttt{name} - название торговца. По умолчанию будет использовано название из \texttt{Mercname.dbf}
\item \texttt{description} - описание торговца. По умолчанию будет использовано описание из \texttt{Mercname.dbf}
\end{itemize}

Пример:\\
Торговец продающий оружие, броню, посохи и свитки на сумму от 1000 до 1500.
В ассортименте также от 1 до 5 эликсиров восстановления.
Охраняется отрядом защитников империи или нейтральных людей

\begin{figure}[H]
\lstinputlisting{docExamples/merchantExample.lua}
\end{figure}