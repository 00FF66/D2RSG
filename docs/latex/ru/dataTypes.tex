\subsection{Типы данных}
В случае с описанием содержимого шаблона генератор понимает и ожидает некоторые заранее определенные таблицы и данные. Поля внутри таких таблиц зависят от того что описывает таблица, другими словами её типа. Ниже будут подробно рассмотрены все существующие типы таблиц, их назначение и поля.

\subsubsection{Расы}
\label{raceTypes}
Расы известные генератору:
\begin{itemize}
\item \texttt{Race.Human} - Защитники Империи
\item \texttt{Race.Undead} - Орды нежити
\item \texttt{Race.Heretic} - Легионы проклатых
\item \texttt{Race.Dwarf} - Горные кланы
\item \texttt{Race.Elf} - Альянс эльфов
\item \texttt{Race.Neutral} - Нейтралы
\end{itemize}
\textbf{\selectlanguage{Russian}Важно:} нейтралы не могут появиться в списке рас функции \texttt{getContents}.

\subsubsection{Субрасы}
\label{subraceTypes}
Субрасы известные генератору:
\begin{itemize}
\item \texttt{Subrace.Human} - Защитники Империи
\item \texttt{Subrace.Undead} - Орды нежити
\item \texttt{Subrace.Heretic} - Легионы проклятых
\item \texttt{Subrace.Dwarf} - Горные кланы
\item \texttt{Subrace.Elf} - Альянс эльфов
\item \texttt{Subrace.Neutral} - Нейтралы
\item \texttt{Subrace.NeutralHuman} - Нейтральные люди
\item \texttt{Subrace.NeutralElf} - Нейтральные эльфы
\item \texttt{Subrace.NeutralGreenSkin} - Зеленокожие
\item \texttt{Subrace.NeutralDragon} - Драконы
\item \texttt{Subrace.NeutralMarsh} - Жители болот
\item \texttt{Subrace.NeutralWater} - Водные обитатели
\item \texttt{Subrace.NeutralBarbarian} - Варвары
\item \texttt{Subrace.NeutralWolf} - Животные
\item \texttt{Subrace.Custom} - Своя субраса
\item \texttt{Subrace.Sub\%id\%} - Дополнительная субраса, где \%id\% это id субрасы в моде (15-34).
\end{itemize}

\subsubsection{Типы предметов}
\label{itemTypes}
Типы предметов известные генератору:
\begin{itemize}
\item \texttt{Item.Armor} - Артефакты (не влияющие на урон)
\item \texttt{Item.Jewel} - Реликвии
\item \texttt{Item.Weapon} - Артефакты (влияющие на урон)
\item \texttt{Item.Banner} - Знамена
\item \texttt{Item.PotionBoost} - Зелья бафов
\item \texttt{Item.PotionHeal} - Зелья лечения
\item \texttt{Item.PotionRevive} - Зелья воскрешения
\item \texttt{Item.PotionPermanent} - Зелья дающие постоянный эффект
\item \texttt{Item.Scroll} - Свитки
\item \texttt{Item.Wand} - Посохи
\item \texttt{Item.Valuable} - Драгоценности
\item \texttt{Item.Orb} - Сферы
\item \texttt{Item.Talisman} - Талисманы
\item \texttt{Item.TravelItem} - Походное снаряжение
\item \texttt{Item.Special} - Особые
\end{itemize}

\subsubsection{Типы заклинаний}
\label{spellTypes}
Типы заклинаний известные генератору:
\begin{itemize}
\item \texttt{Spell.Attack} - Наносящие урон, например 'Армагеддон'
\item \texttt{Spell.Lower} - Снижающие показатели (дебафы), например 'Гниение'
\item \texttt{Spell.Heal} - Лечащие заклинания, например 'Исцеление'
\item \texttt{Spell.Boost} - Увеличивающие показатели (бафы), например 'Гимн Кланов'
\item \texttt{Spell.Summon} - Призыв, например 'Вызов Белиарха'
\item \texttt{Spell.Fog} - Наложение тумана войны, например 'Сумерки'
\item \texttt{Spell.Unfog} - Открытие тумана войны, например 'Свет дня'
\item \texttt{Spell.RestoreMove} - Восстановление очков движения, например 'Подвижность'
\item \texttt{Spell.Invisibility} - Невидимость, например 'Сокрытие'
\item \texttt{Spell.RemoveRod} - Удаление жезлов, например 'Водосточный колодец'
\item \texttt{Spell.ChangeTerrain} - Смена типа поверхности, например 'Создание Рощи'
\item \texttt{Spell.GiveWards} - Наложение даров (вардов), например 'Влиятельность' (Мотлин), 'Поспешность' (MNS)
\end{itemize}

\subsubsection{Типы границ зон}
\label{borderTypes}
Типы границ зон известные генератору:
\begin{itemize}
\item \texttt{Border.Open} - Зона не имеет закрытых границ
\item \texttt{Border.SemiOpen} - Границы зоны содержат несколько непроходимых участков, разделенных проходами
\item \texttt{Border.Cloded} - Зона окружена непроходимой границей, для доступа внутрь необходим \hyperref[connection]{\selectlanguage{Russian}Проход}
\item \texttt{Border.Water} - Зона окружена водой
\end{itemize}

\subsubsection{Типы ресурсов}
\label{resourceTypes}
Типы ресурсов известные генератору:
\begin{itemize}
\item \texttt{Resource.Gold} - Золото
\item \texttt{Resource.InfernalMana} - Мана преисподней
\item \texttt{Resource.LifeMana} - Мана жизни
\item \texttt{Resource.DeathMana} - Мана смерти
\item \texttt{Resource.RunicMana} - Мана рун
\item \texttt{Resource.GroveMana} - Мана рощи
\end{itemize}