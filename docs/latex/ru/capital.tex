\subsubsection{Столица}
\label{capital}
Описывает столицу в стартовой зоне под контролем игрока. Из-за особенностей игры, описание столицы совпадает с описанием городов лишь частично.

Обязательных полей нет, пустая таблица создаст столицу со стражем, но без стартовых предметов. Стартовые зоны игроков всегда будут содержать столицы в своих центрах.

Необязательные поля:
\begin{itemize}
\item \texttt{garrison} - защитники в гарнизоне столицы. См. \hyperref[group]{\selectlanguage{Russian}Группа / Отряд}. \textbf{\selectlanguage{Russian}Важно:} независимо от ценности и субрас гарнизона в нем всегда будет создан страж столицы соответствующий расе игрока. \textbf{\selectlanguage{Russian}Важно:} Предметы указанные для защитников будут созданы в инвентаре столицы
\item \texttt{spells} - список идентификаторов заклинаний из \texttt{GSpells.dbf} которые будут известны игроку со старта
\item \texttt{aiPriority} - приоритет столицы для ИИ в диапазоне \texttt{[0 : 6]}
\item \texttt{name} - название столицы. По умолчанию будет использовано название из \texttt{Cityname.dbf}
\end{itemize}

Пример:\\
Столица с гарнизоном из войск империи.
В инвентаре будут созданы зелья лечения и воскрешения на 750 - 800 ценности.
Игроку-владельцу столицы со старта будут известны 2 заклинания

\begin{figure}[H]
\lstinputlisting{docExamples/capitalExample.lua}
\end{figure}