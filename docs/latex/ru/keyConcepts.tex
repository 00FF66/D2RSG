\section{Ключевые понятия}
Зоны, их содержимое, а также соединения между зонами описываются с помощью Lua таблиц.
Простейшая таблица в языке Lua выглядит следующим образом:\\
\begin{lstlisting}
{ }
\end{lstlisting}
Таблицы могут быть пустыми, либо содержать именованые поля, не именованые поля и другие таблицы:
Пустая таблица:\\
\begin{lstlisting}
{ }
\end{lstlisting}
Таблица с именоваными полями:\\
\begin{lstlisting}
{ min = 0, max = 100 }
\end{lstlisting}
Таблица с полями без имен, используются индексы (начиная с 1):\\
\begin{lstlisting}
{ [1] = 1, [2] = 100 }
\end{lstlisting}
Таблица с двумя вложенными таблицами:\\
\begin{lstlisting}
{ { min = 0, max = 100 }, { min = 10, max = 50 } }
\end{lstlisting}
Для удобства чтения таблицы могут быть сохранены в переменные, тем самым получая имя:
\begin{lstlisting}
local emptyTable = {}
\end{lstlisting}


Вложенность таблиц и их полей важна, а порядок задания полей не играет роли.\\
Таким образом приведенный выше пример шаблона содержит таблицу \texttt{template} с полями
\texttt{name}, \texttt{description}, \texttt{roads}, \texttt{forest} и другими.
Также, поле \texttt{getContents} в таблице \texttt{template} указывает на функцию которая возвращает безымянную таблицу 
содержащую 2 именованых поля \texttt{zones} и \texttt{connections}, которые в свою очередь также являются таблицами.
Для удобства чтения этот блок кода можно было записать иначе:
\begin{figure}[h]
\lstinputlisting{docExamples/getContents.lua}
\caption{\selectlanguage{Russian}Вариант написания функции \texttt{getContents}}
\end{figure}

\newpage
% Describe common data types and enumerations used by generator
\subsection{Типы данных}
В случае с описанием содержимого шаблона генератор понимает и ожидает некоторые заранее определенные таблицы и данные. Поля внутри таких таблиц зависят от того что описывает таблица, другими словами её типа. Ниже будут подробно рассмотрены все существующие типы таблиц, их назначение и поля.

\subsubsection{Расы}
\label{raceTypes}
Расы известные генератору:
\begin{itemize}
\item \texttt{Race.Human} - Защитники Империи
\item \texttt{Race.Undead} - Орды нежити
\item \texttt{Race.Heretic} - Легионы проклатых
\item \texttt{Race.Dwarf} - Горные кланы
\item \texttt{Race.Elf} - Альянс эльфов
\item \texttt{Race.Neutral} - Нейтралы
\end{itemize}
\textbf{\selectlanguage{Russian}Важно:} нейтралы не могут появиться в списке рас функции \texttt{getContents}.

\subsubsection{Субрасы}
\label{subraceTypes}
Субрасы известные генератору:
\begin{itemize}
\item \texttt{Subrace.Human} - Защитники Империи
\item \texttt{Subrace.Undead} - Орды нежити
\item \texttt{Subrace.Heretic} - Легионы проклятых
\item \texttt{Subrace.Dwarf} - Горные кланы
\item \texttt{Subrace.Elf} - Альянс эльфов
\item \texttt{Subrace.Neutral} - Нейтралы
\item \texttt{Subrace.NeutralHuman} - Нейтральные люди
\item \texttt{Subrace.NeutralElf} - Нейтральные эльфы
\item \texttt{Subrace.NeutralGreenSkin} - Зеленокожие
\item \texttt{Subrace.NeutralDragon} - Драконы
\item \texttt{Subrace.NeutralMarsh} - Жители болот
\item \texttt{Subrace.NeutralWater} - Водные обитатели
\item \texttt{Subrace.NeutralBarbarian} - Варвары
\item \texttt{Subrace.NeutralWolf} - Животные
\item \texttt{Subrace.Custom} - Своя субраса
\end{itemize}

\subsubsection{Типы предметов}
\label{itemTypes}
Типы предметов известные генератору:
\begin{itemize}
\item \texttt{Item.Armor} - Артефакты (не влияющие на урон)
\item \texttt{Item.Jewel} - Реликвии
\item \texttt{Item.Weapon} - Артефакты (влияющие на урон)
\item \texttt{Item.Banner} - Знамена
\item \texttt{Item.PotionBoost} - Зелья бафов
\item \texttt{Item.PotionHeal} - Зелья лечения
\item \texttt{Item.PotionRevive} - Зелья воскрешения
\item \texttt{Item.PotionPermanent} - Зелья дающие постоянный эффект
\item \texttt{Item.Scroll} - Свитки
\item \texttt{Item.Wand} - Посохи
\item \texttt{Item.Valuable} - Драгоценности
\item \texttt{Item.Orb} - Сферы
\item \texttt{Item.Talisman} - Талисманы
\item \texttt{Item.TravelItem} - Походное снаряжение
\item \texttt{Item.Special} - Особые
\end{itemize}

\subsubsection{Типы заклинаний}
\label{spellTypes}
Типы заклинаний известные генератору:
\begin{itemize}
\item \texttt{Spell.Attack} - Наносящие урон, например 'Армагеддон'
\item \texttt{Spell.Lower} - Снижающие показатели (дебафы), например 'Гниение'
\item \texttt{Spell.Heal} - Лечащие заклинания, например 'Исцеление'
\item \texttt{Spell.Boost} - Увеличивающие показатели (бафы), например 'Гимн Кланов'
\item \texttt{Spell.Summon} - Призыв, например 'Вызов Белиарха'
\item \texttt{Spell.Fog} - Наложение тумана войны, например 'Сумерки'
\item \texttt{Spell.Unfog} - Открытие тумана войны, например 'Свет дня'
\item \texttt{Spell.RestoreMove} - Восстановление очков движения, например 'Подвижность'
\item \texttt{Spell.Invisibility} - Невидимость, например 'Сокрытие'
\item \texttt{Spell.RemoveRod} - Удаление жезлов, например 'Водосточный колодец'
\item \texttt{Spell.ChangeTerrain} - Смена типа поверхности, например 'Создание Рощи'
\item \texttt{Spell.GiveWards} - Наложение даров (вардов), например 'Влиятельность' (Мотлин), 'Поспешность' (MNS)
\end{itemize}
\newpage
% Describe zone types and their behavior
\subsection{Типы зон}
\label{zoneTypes}
Типы зон поддерживаемые генератором:
\begin{itemize}
\item \texttt{Zone.PlayerStart} - стартовая локация игрока. В центре зоны появится столица указанной расы. Первый рудник золота и источник родной маны будут под контролем расы.
\item \texttt{Zone.Treasure} - сокровищница или обычная зона. В зоне будет создано множесто случайных проходов между объектами, выходами, а также ведущие в тупик.
\item \texttt{Zone.Junction} - проходная, соединительная зона. Будет создан один проход соединяющий выходы.
\end{itemize}
\newpage
% Describe all supported tables and their fields
\subsection{Типы таблиц}

% Zone description
\subsubsection{Зона}
\label{zone}
Определяет зону на карте сценария.

Обязательные поля:
\begin{itemize}
\item \texttt{id} - идентификатор зоны. Число, уникально определяющее зону
\item \texttt{type} - тип зоны. См. \hyperref[zoneTypes]{\selectlanguage{Russian}Типы зон}
\item \texttt{size} - относительный размер зоны
\end{itemize}

В случае если в качестве типа зоны выбрана стартовая локация \texttt{type = Zone.PlayerStart}, нужно указать уникальную расу для игрока в этой зоне:
\begin{itemize}
\item \texttt{race} - раса игрока. Одна из поданного в \texttt{getContents} списка рас или заранее заданная. См. \hyperref[raceTypes]{\selectlanguage{Russian}Расы}
\end{itemize}

Необязательные поля:
\begin{itemize}
\item \texttt{mines} - источники ресурсов в зоне. См. \hyperref[crystals]{\selectlanguage{Russian}Источники ресурсов}
\item \texttt{towns} - список нейтральных городов. См. \hyperref[city]{\selectlanguage{Russian}Город}
\item \texttt{ruins} - список руин. См. \hyperref[ruin]{\selectlanguage{Russian}Руины}
\item \texttt{merchants} - список торговцев. См. \hyperref[merchant]{\selectlanguage{Russian}Торговец}
\item \texttt{mages} - список башен магов. См. \hyperref[mage]{\selectlanguage{Russian}Башня мага}
\item \texttt{mercenaries} - список лагерей наемников. См. \hyperref[mercenary]{\selectlanguage{Russian}Лагерь наемников}
\item \texttt{trainers} - список тренировочных лагерей. См. \hyperref[trainer]{\selectlanguage{Russian}Тренер}
\item \texttt{stacks} - список групп нейтральных отрядов в зоне. См. \hyperref[neutralStacks]{\selectlanguage{Russian}Нейтральные отряды}
\item \texttt{bags} - неохраняемые сундуки в зоне. См. \hyperref[bags]{\selectlanguage{Russian}Неохраняемые сундуки}
\end{itemize}

Для стартовых локаций игроков также возможно задание гарнизона, предметов в столице и заклинаний известных игроку со старта:
\begin{itemize}
\item \texttt{capital} - столица игрока. См. \hyperref[capital]{\selectlanguage{Russian}Столица}
\end{itemize}

Примеры:\\
Стартовая зона с относительным размером 20. Здесь будет столица первого игрока из списка рас.
В зоне будут созданы 1 золотой рудник и 1 источник маны рун.

\begin{figure}[H]
\lstinputlisting{docExamples/zoneExample1.lua}
\end{figure}

Стартовая зона с относительным размером 20. Здесь будет столица первого игрока из списка рас.
Независимо от выбранной расы, в гарнизоне столице будут созданы юниты империи общей ценностью 150 - 175.
В инвентаре столицы будут созданы зелья лечения и воскрешения ценностью 750 - 800.
Игроку-владельцу со старта будут известны заклинания 'Ледяной щит' и 'Молния'.

\begin{figure}[H]
\lstinputlisting{docExamples/zoneExample2.lua}
\end{figure}

Обычная зона (сокровищница) с относительным размером 35.
В зоне будут случайно расположены 12 нейтральных отрядов общей ценностью 1200 - 1300,
а также 5 нейтральных отрядов общей ценностью 2000.

\begin{figure}[H]
\lstinputlisting{docExamples/zoneExample3.lua}
\end{figure}
\newpage
% Connection description
\subsubsection{Проход}
\label{connection}
Определяет соединение двух зон, проход между ними.
Между двумя зонами может быть более одного прохода.

Обязательные поля:
\begin{itemize}
\item \texttt{from} - идентификатор первой зоны. См. \hyperref[zone]{\selectlanguage{Russian}Зона}
\item \texttt{to} - идентификатор второй зоны. См. \hyperref[zone]{\selectlanguage{Russian}Зона}
\end{itemize}

Необязательные поля:
\begin{itemize}
\item \texttt{guard} - отряд охраняющий проход между зонами. См. \hyperref[group]{\selectlanguage{Russian}Группа / Отряд}
\end{itemize}

Примеры:\\
Проход между зонами 0 и 1, без охраны
\begin{figure}[h]
\lstinputlisting{docExamples/connectionExample1.lua}
\end{figure}

Проход между зонами 1 и 0, аналогичен предыдущему примеру
\begin{figure}[h]
\lstinputlisting{docExamples/connectionExample2.lua}
\end{figure}

Проход между зонами 0 и 3.
Охраняется отрядом ценностью 1750 - 2250.
В инвентаре отряда будут созданы артефакты, знамена или перманентные зелья общей ценностью 2500 - 3000
\begin{figure}[h]
\lstinputlisting{docExamples/connectionExample3.lua}
\end{figure}
\newpage
% Diplomacy relations description
\subsubsection{Дипломатия}
\label{diplomacy}
Определяет дипломатические отношения между двумя расами.

Обязательные поля:
\begin{itemize}
\item \texttt{raceA} - первая раса. См. \hyperref[raceTypes]{\selectlanguage{Russian}Раса}
\item \texttt{raceB} - вторая раса. См. \hyperref[raceTypes]{\selectlanguage{Russian}Раса}
\item \texttt{relation} - уровень отношений в диапазоне \texttt{[0 : 100]}. 0 - вражда, 100 - мир
\end{itemize}
Необязательные поля:
\begin{itemize}
\item \texttt{alliance} - \texttt{true} если расы находятся в союзе. Не может быть \texttt{true} когда \texttt{alwaysAtWar} также \texttt{true}. По умолчанию \texttt{false}
\item \texttt{alwaysAtWar} - \texttt{true} если расы находятся в состоянии вечной вражды. Не может быть \texttt{true} когда \texttt{alliance} также \texttt{true}. По умолчанию \texttt{false}
\item \texttt{permanentAlliance} - \texttt{true} если альянс считается постоянным для ИИ. Не может быть \texttt{true} когда \texttt{alliance} \texttt{false}. По умолчанию \texttt{false}
\end{itemize}

Примеры:\\
Две расы находятся в состоянии вражды:

\begin{figure}[H]
\lstinputlisting{docExamples/diplomacyExample1.lua}
\end{figure}

Две расы находятся в состоянии вражды, заключение союза невозможно:

\begin{figure}[H]
\lstinputlisting{docExamples/diplomacyExample2.lua}
\end{figure}

Две расы находятся в союзе. ИИ не может расторгнуть союз:

\begin{figure}[H]
\lstinputlisting{docExamples/diplomacyExample3.lua}
\end{figure}
\newpage
% Value description
\subsubsection{Ценность}
\label{value}
Задает ценность группы (отряда), юнита, предмета или заклинания в диапазоне \texttt{[min : max]}.

Расчет ценности генератором сценариев:
\begin{itemize}
\item ценность групп, отрядов и юнитов определяется их опытом за убийство, значение \texttt{XP\_KILLED} из GUnits.dbf
\item ценность предметов определяется как сумма всех ресурсов в стоимости их покупки у торговца, значение \texttt{VALUE} из GItems.dbf
\item ценность заклинаний определяется как сумма всех ресурсов в стоимости их покупки в башне мага, значение \texttt{BUY\_C} из GSpells.dbf
\end{itemize}

Сумма всех ресурсов:\\
К примеру предмет \texttt{А} стоит 100 золота и 50 маны жизни. Сумма всех ресурсов его стоимости даст ценность равную 150.
Предмет \texttt{Б} стоит 150 золота. Его ценность, также как у \texttt{А}, будет 150.\\
Для генератора золото и все типы маны равноценны.

Обязательные поля:
\begin{itemize}
\item \texttt{min} - минимальное значение ценности. Целое число
\item \texttt{max} - максимальное значение ценности. Целое число
\end{itemize}

Примеры:\\
Ценность от 100 до 200 включительно:\\
\begin{lstlisting}
{ min = 100, max = 200 }
\end{lstlisting}
Ценность 175:\\
\begin{lstlisting}
{ min = 175, max = 175 }
\end{lstlisting}
\newpage
% Loot item description
\subsubsection{Предмет награды}
\label{item}
Описывает обязательный предмет в награде.

Обязательные поля:
\begin{itemize}
\item \texttt{id} - идентификатор предмета из GItems.dbf
\item \texttt{min} - минимальное число предметов
\item \texttt{max} - максимальное число предметов
\end{itemize}

Пример:\\
От 1 до 5 эликсиров восстановления\\
\begin{lstlisting}
{ id = 'g000ig0006', min = 1, max = 5 }
\end{lstlisting}
\newpage
% Loot description
\subsubsection{Награда}
\label{loot}
Описывает награду (лут). Награда может состоять из случайных и обязательных предметов.

Обязательных полей нет, пустая таблица создаст пустую награду.

Необязательные поля:
\begin{itemize}
\item \texttt{value} - ценность случайных предметов в награде. См. \hyperref[value]{\selectlanguage{Russian}Ценность}
\item \texttt{itemValue} - диапазон ценности каждого случайного предмета в награде. См. \hyperref[value]{\selectlanguage{Russian}Ценность}
\item \texttt{itemTypes} - список типов случайных предметов которые могут быть созданы. В случае пустого списка может быть создан предмет любого типа, кроме специального \texttt{Item.Special}
\item \texttt{items} - список предметов которые обязательно должны быть созданы. Не зависит от ценности награды. См. \hyperref[item]{\selectlanguage{Russian}Предмет награды}
\end{itemize}

Примеры:\\
Случайная награда из зелий-бафов и воскрешения общей стоимостью от 500 до 750,
а также от 3 до 5 эликсиров восстановления

\begin{figure}[H]
\lstinputlisting{docExamples/lootExample1.lua}
\end{figure}

Награда из 5 зелий восстановления

\begin{figure}[H]
\lstinputlisting{docExamples/lootExample2.lua}
\end{figure}

Случайная награда из знамен и сапог общей стоимостью от 2500 до 2750

\begin{figure}[H]
\lstinputlisting{docExamples/lootExample3.lua}
\end{figure}

Случайная награда общей стоимостью от 1600 до 1800 
из зелий лечения или воскрешения каждое стоимостью от 200 до 450. 
Подняв минимальную ценность каждой случайной вещи до 200 
мы исключаем дешевые зелья лечения (50 hp) из награды.

\begin{figure}[H]
\lstinputlisting{docExamples/lootExample4.lua}
\end{figure}
\newpage
% Group / Stack description
\subsubsection{Группа / Отряд}
\label{group}
Описывает группу юнитов в руинах или гарнизоне города, либо отряд с лидером.

Обязательные поля:
\begin{itemize}
\item \texttt{value} - общая ценность юнитов. См. \hyperref[value]{\selectlanguage{Russian}Ценность}
\end{itemize}

Необязательные поля:
\begin{itemize}
\item \texttt{subraceTypes} - список сабрас определяющий какие типы юнитов могут быть созданы. См. \hyperref[subraceTypes]{\selectlanguage{Russian}Субрасы}. В случае пустого списка могут быть созданы юниты любых субрас
\item \texttt{loot} - награда. См. \hyperref[loot]{\selectlanguage{Russian}Награда}. В случае отряда награда создаст предметы инвентаря. Для группы в гарнизоне города награда создаст предметы в городе. \textbf{\selectlanguage{Russian}Важно:} награда группы внутри руин \textbf{\selectlanguage{Russian}не учитывается!}. Награда руин задается отдельно. См. \hyperref[ruin]{\selectlanguage{Russian}Руины}
\end{itemize}

Пример:\\
Группа / Отряд из гномов или нейтральных людей.
В качестве награды будут созданы зелья лечения и бафов на 200 - 300 ценности.

\begin{figure}[H]
\lstinputlisting{docExamples/groupExample.lua}
\end{figure}
\newpage
% City description
\subsubsection{Город}
\label{city}
Описывает нейтральный город, его гарнизон и посещающий отряд, а также предметы внутри города.

Обязательные поля:
\begin{itemize}
\item \texttt{tier} - уровень города в диапазоне \texttt{[1: 5]}
\end{itemize}

Необязательные поля:
\begin{itemize}
\item \texttt{garrison} - защитники в гарнизоне города. См. \hyperref[group]{\selectlanguage{Russian}Группа / Отряд}. \textbf{\selectlanguage{Russian}Важно:} Предметы указанные для защитников будут созданы в инвентаре города.
\item \texttt{stack} - отряд посещающий город. См. \hyperref[group]{\selectlanguage{Russian}Группа / Отряд}
\end{itemize}

Пример:\\
Город 1 уровня с гарнизоном из гномов или нейтральных людей.
В гарнизоне будут созданы зелья лечения и бафов на 200 - 300 ценности.
Посещающий отряд в воротах города состоит из варваров и волков.

\begin{figure}[H]
\lstinputlisting{docExamples/cityExample.lua}
\end{figure}
\newpage
% Capital description
\subsubsection{Столица}
\label{capital}
Описывает столицу в стартовой зоне под контролем игрока. Из-за особенностей игры, описание столицы совпадает с описанием городов лишь частично.

Обязательных полей нет, пустая таблица создаст столицу со стражем, но без стартовых предметов. Стартовые зоны игроков всегда будут содержать столицы в своих центрах.

Необязательные поля:
\begin{itemize}
\item \texttt{garrison} - защитники в гарнизоне столицы. См. \hyperref[group]{\selectlanguage{Russian}Группа / Отряд}. \textbf{\selectlanguage{Russian}Важно:} независимо от ценности и субрас гарнизона в нем всегда будет создан страж столицы соответствующий расе игрока. \textbf{\selectlanguage{Russian}Важно:} Предметы указанные для защитников будут созданы в инвентаре столицы
\item \texttt{spells} - список идентификаторов заклинаний из \texttt{GSpells.dbf} которые будут известны игроку со старта
\item \texttt{aiPriority} - приоритет столицы для ИИ в диапазоне \texttt{[0 : 6]}
\item \texttt{name} - название столицы. По умолчанию будет использовано название из \texttt{Cityname.dbf}
\item \texttt{guardian} - определяет будет ли сгенерирован страж столицы. По умолчанию true
\item \texttt{gapMask} - блокирует появление объектов и гор с определенных сторон. См. \hyperref[gapMask]{\selectlanguage{Russian}Маска отступа}
\end{itemize}

Пример:\\
Столица с гарнизоном из войск империи.
В инвентаре будут созданы зелья лечения и воскрешения на 750 - 800 ценности.
Игроку-владельцу столицы со старта будут известны 2 заклинания

\begin{figure}[H]
\lstinputlisting{docExamples/capitalExample.lua}
\end{figure}
\newpage
% Ruins description
\subsubsection{Руины}
\label{ruin}
Описывает руины, их защитников и награду.

Обязательные поля:
\begin{itemize}
\item \texttt{guard} - юниты внутри руин. См. \hyperref[group]{\selectlanguage{Russian}Группа / Отряд}. \textbf{\selectlanguage{Russian}Важно:} предметы указанные для этой группы игнорируются!
\end{itemize}

Необязательные поля:
\begin{itemize}
\item \texttt{gold} - награда за руины в золоте. См. \hyperref[value]{\selectlanguage{Russian}Ценность}
\item \texttt{loot} - предмет-награда за руины. См. \hyperref[loot]{\selectlanguage{Russian}Награда}. \textbf{\selectlanguage{Russian}Важно:} только один предмет из награды будет создан. Обязательные предметы имеют приоритет над случайными. В случае обязательных предметов будет выбран первый из списка.
\item \texttt{aiPriority} - приоритет руин для ИИ в диапазоне \texttt{[0 : 6]}
\end{itemize}

Пример:\\
Руины с наградой из эликсира восстановления и 200 - 250 золота.
Охраняются группой защитников империи или нейтральных людей

\begin{figure}[H]
\lstinputlisting{docExamples/ruinExample.lua}
\end{figure}
\newpage
% Merchant description
\subsubsection{Торговец}
\label{merchant}
Описывает лавку торговца, его товары и потенциальную охрану.

Обязательные поля:
\begin{itemize}
\item \texttt{goods} - ассортимент предметов торговца. См. \hyperref[loot]{\selectlanguage{Russian}Награда}. Торговцы в игре не продают драгоценности (\texttt{Item.Valuable}), поэтому генератор \textbf{\selectlanguage{Russian}никогда} не создаст предметы этого типа среди случайных. Если такие предметы по какой-то причине все же необходимы в ассортименте торговца, следует указать их в списке обязательных предметов
\end{itemize}

Необязательные поля:
\begin{itemize}
\item \texttt{guard} - отряд охраняющий вход в лавку торговца. См. \hyperref[group]{\selectlanguage{Russian}Группа / Отряд}
\end{itemize}

Пример:\\
Торговец продающий оружие, броню, посохи и свитки на сумму от 1000 до 1500.
В ассортименте также от 1 до 5 эликсиров восстановления.
Охраняется отрядом защитников империи или нейтральных людей

\begin{figure}[H]
\lstinputlisting{docExamples/merchantExample.lua}
\end{figure}
\newpage
% Mage description
\subsubsection{Башня мага}
\label{mage}
Описывает башню мага, его заклинания и потенциальную охрану. Заклинания делятся на случайные и обязательные.

Обязательных полей нет, пустая таблица описывает башню мага без заклинаний и охраны.

Необязательные поля:
\begin{itemize}
\item \texttt{value} - общая ценность случайных заклинаний. Ценностью заклинания считается стоимость его покупки (сумма всех ресурсов), поле \texttt{BUY\_C} из GSpells.dbf.  См. \hyperref[value]{\selectlanguage{Russian}Ценность}
\item \texttt{spellLevel} - диапазон уровней каждого случайного заклинания. Допустимые значения: \texttt{[1 : 5]}. Если не задано, в башне могут продаваться заклинания любых уровней
\item \texttt{spellTypes} - список типов случайных заклинаний которые могут появиться. См. \hyperref[spellTypes]{\selectlanguage{Russian}Типы заклинаний}. В случае пустого списка могут быть выбраны заклинания любых типов
\item \texttt{spells} - список идентификаторов заклинаний из GSpells.dbf которые обязательно появятся в башне мага.
\item \texttt{guard} - отряд охраняющий вход в башню мага. См. \hyperref[group]{\selectlanguage{Russian}Группа / Отряд}
\item \texttt{aiPriority} - приоритет башни мага для ИИ в диапазоне \texttt{[0 : 6]}
\end{itemize}

Пример:\\
Башня мага с заклинаниями лечения и ускорений общей ценностью от 200 до 1000.
В ассортименте обязательно появятся 'Быстрота' и 'Молния'

\begin{figure}[H]
\lstinputlisting{docExamples/mageExample1.lua}
\end{figure}

Башня мага с заклинаниями только 2 уровня общей ценностью от 2700 до 3000.
Если каждое заклинание 2 уровня стоит порядка 300 золота (ценность 300),
получим примерно 9-10 заклинаний в ассортименте

\begin{figure}[H]
\lstinputlisting{docExamples/mageExample2.lua}
\end{figure}
\newpage
% Mercenary unit description
\subsubsection{Наемник}
\label{mercenaryUnit}
Описывает обязательного наемника в лагере.

Обязательные поля:
\begin{itemize}
\item \texttt{id} - идентификатор юнита из GUnits.dbf
\item \texttt{level} - уровень юнита. \textbf{\selectlanguage{Russian}Важно:} уровень юнита не может быть ниже базового!
\end{itemize}

Необязательные поля:
\begin{itemize}
\item \texttt{unique} - определяет будет ли юнит доступен для найма единожды
\end{itemize}

Пример:\\
'Рейнджер' (т2 лучник империи), доступный для найма бесконечное число раз

\begin{lstlisting}
{ id = 'g000uu0007', level = 2, unique = false }
\end{lstlisting}

'Скелет-воин' (т3 воин нежити) 20 уровня, доступный для найма единожды

\begin{lstlisting}
{ id = 'g000uu0088', level = 20, unique = true }
\end{lstlisting}
\newpage
% Mercenary camp description
\subsubsection{Лагерь наемников}
\label{mercenary}
Описывает лагерь наемников, юнитов в нем, а также потенциальную охрану. Юниты делятся на случайные и обязательные.

Обязательные поля:
\begin{itemize}
\item \texttt{value} - общая ценность случайных юнитов. См. \hyperref[value]{\selectlanguage{Russian}Ценность}
\end{itemize}

Необязательные поля:
\begin{itemize}
\item \texttt{subraceTypes} - список сабрас определяющий какие типы юнитов могут быть созданы. См. \hyperref[subraceTypes]{\selectlanguage{Russian}Субрасы}. В случае пустого списка могут быть созданы юниты любых субрас
\item \texttt{units} - список наемников которые обязательно должны быть в лагере. Не зависит от ценности случайных юнитов. См. \hyperref[mercenaryUnit]{\selectlanguage{Russian}Наемник}
\item \texttt{guard} - отряд охраняющий вход в лагерь. См. \hyperref[group]{\selectlanguage{Russian}Группа / Отряд}
\end{itemize}

Пример:\\
Лагерь наемников с юнитами империи и нейтральными людьми общей ценностью от 200 до 250.
В лагере обязательно будут доступны 'Рейнджер' и 'Скелет-воин' 20 уровня, доступный для найма лишь единожды

\begin{figure}[H]
\lstinputlisting{docExamples/mercenaryExample.lua}
\end{figure}
\newpage
% Trainer description
\subsubsection{Тренер}
\label{trainer}
Описывает тренировочный лагерь и его опциональную охрану.

Обязательных полей нет. Пустая таблица описывает тренировочный лагерь без охраны.

Необязательные поля:
\begin{itemize}
\item \texttt{guard} - отряд охраняющий вход в лагерь. См. \hyperref[group]{\selectlanguage{Russian}Группа / Отряд}
\item \texttt{aiPriority} - приоритет тренера для ИИ в диапазоне \texttt{[0 : 6]}
\item \texttt{name} - название тренера. По умолчанию будет использовано название из \texttt{Trainame.dbf}
\item \texttt{description} - описание тренера. По умолчанию будет использовано описание из \texttt{Trainame.dbf}
\end{itemize}

Пример:\\
Тренировочный лагерь с отрядом охраны из людей и нейтральных людей общей ценностью 200 - 250.

\begin{figure}[H]
\lstinputlisting{docExamples/trainerExample.lua}
\end{figure}
\newpage
% Tradable resource description
\subsubsection{Ресурс рынка}
\label{marketresource}
Описывает один из ресурсов доступный для обмена на рынке.

Обязательные поля:
\begin{itemize}
\item \texttt{resource} - тип ресурса. См. \hyperref[resourceTypes]{\selectlanguage{Russian}типы ресурсов}
\end{itemize}

Необязательные поля:
\begin{itemize}
\item \texttt{value} - количество ресурса, если его запасы конечны. См. \hyperref[value]{\selectlanguage{Russian}Ценность}.
\item \texttt{infinite} - исчерпаемость ресурса. При значении \texttt{true} ресурс бесконечен, а поле \texttt{value} игнорируется.
\end{itemize}

Пример:\\
Золото, бесконечный запас

\begin{lstlisting}
{ resource = Resource.Gold, infinite = true }
\end{lstlisting}

Мана жизни, количество от 200 до 300

\begin{lstlisting}
{ resource = Resource.LifeMana, value = { min = 200, max = 300 } }
\end{lstlisting}
\newpage
% Resource market description
\subsubsection{Рынок ресурсов}
\label{resourcemarket}
Описывает рынок, его обменный курс, запасы ресурсов и потенциальную охрану.

Обязательных полей нет, пустая таблица описывает рынок с обменным курсом по-умолчанию, без ресурсов и охраны.

Необязательные поля:
\begin{itemize}
\item \texttt{exchangeRates} - строка, содержащая скрипт обменного курса, специфичного для рынка.
\item \texttt{stock} - список \hyperref[marketresource]{\selectlanguage{Russian}ресурсов рынка}.
\item \texttt{guard} - отряд охраняющий вход на рынок. См. \hyperref[group]{\selectlanguage{Russian}Группа / Отряд}
\item \texttt{aiPriority} - приоритет рынка для ИИ в диапазоне \texttt{[0 : 6]}
\item \texttt{name} - название рынка. По умолчанию будет использовано название из \texttt{Marketname.dbf}
\item \texttt{description} - описание рынка. По умолчанию будет использовано описание из \texttt{Marketname.dbf}
\end{itemize}

Пример:\\
Рынок с бесконечным запасом золота, 100 единицами маны жизни, 200 - 300 единиц маны преисподней и 500 - 550 единиц маны смерти

\begin{figure}[H]
\lstinputlisting{docExamples/marketExample1.lua}
\end{figure}

Рынок с бесконечным запасом маны и специальным обменным курсом который позволяет игрокам менять золото на любой вид маны в соотношении 10 к 1

\begin{figure}[H]
\lstinputlisting{docExamples/marketExample2.lua}
\end{figure}
\newpage
% Crystals description
\subsubsection{Источники ресурсов}
\label{crystals}
Описывает количество источников ресурсов каждого типа внутри зоны. В случае стартовых зон, первый источник золота и родной для игрока маны будут автоматически под владением игрока этой зоны.

Обязательных полей нет. Пустая таблица описывает зону без источников ресурсов.

Необязательные поля:
\begin{itemize}
\item \texttt{gold} - количество золотых шахт 
\item \texttt{infernalMana} - количество источников маны преисподней
\item \texttt{lifeMana} - количество источников маны жизни
\item \texttt{deathMana} - количество источников маны смерти
\item \texttt{runicMana} - количество источников маны рун
\item \texttt{groveMana} - количество источников лесного эликсира
\end{itemize}

Пример:\\
Зона с тремя золотыми шахтами и одним источником маны каждого вида

\begin{figure}[H]
\lstinputlisting{docExamples/crystalsExample.lua}
\end{figure}
\newpage
% Neutral stacks description
\subsubsection{Нейтральные отряды}
\label{neutralStacks}
Описывает группу из нескольких нейтральных отрядов создаваемых по общему для них описанию.
Нейтральные отряды будут расположены случайным образом внутри зоны и могут блокировать проходы, а также доступ к некоторым объектам.

Обязательных полей нет. Пустая таблица описывает пустую группу нейтральных отрядов.

Необязательные поля:
\begin{itemize}
\item Все поля, описывающие отряд. Ценность и награды будут равномерно распределены по всем отрядам в группе. См. \hyperref[group]{\selectlanguage{Russian}Группа / Отряд}
\item \texttt{count} - количество случайных отрядов в группе
\item \texttt{owner} - раса, контролирующая отряды. По умолчанию отряды принадлежат нейтральной расе. См. \hyperref[raceTypes]{\selectlanguage{Russian}Расы}
\end{itemize}

Примеры:\\
12 случайных отрядов общей ценностью 1200.
Награда из зелий лечения и буста общей ценностью 200 - 300.
На 12 отрядов обязательно будут созданы  1 - 5 эликсиров восстановления.

\begin{figure}[H]
\lstinputlisting{docExamples/neutralStacksExample1.lua}
\end{figure}

5 случайных отрядов из воинов империи и нейтральных людей общей ценностью 1500 - 1600.

\begin{figure}[H]
\lstinputlisting{docExamples/neutralStacksExample2.lua}
\end{figure}
\newpage
% Bags description
\subsubsection{Неохраняемые сундуки}
\label{bags}
Описывает неохраняемые сундуки внутри зоны.

Обязательных полей нет. Пустая таблица описывает зону без сундуков.

Необязательные поля:
\begin{itemize}
\item \texttt{loot} - общая награда всех сундуков. Будет равномерно распределена по всем сундукам. См. \hyperref[loot]{\selectlanguage{Russian}Награда}
\item \texttt{count} - количество сундуков в зоне
\item \texttt{aiPriority} - приоритет сундуков для ИИ в диапазоне \texttt{[0 : 6]}
\end{itemize}

Пример:\\
10 сундуков с зельями на буст и воскрешение общей суммой 5000 - 7500.
На 10 сундуков обязательно будут созданы 15 эликсиров восстановления.

\begin{figure}[H]
\lstinputlisting{docExamples/bagsExample.lua}
\end{figure}
\newpage
