\subsubsection{Нейтральные отряды}
\label{neutralStacks}
Описывает группу из нескольких нейтральных отрядов создаваемых по общему для них описанию.
Нейтральные отряды будут расположены случайным образом внутри зоны и могут блокировать проходы, а также доступ к некоторым объектам.

Обязательных полей нет. Пустая таблица описывает пустую группу нейтральных отрядов.

Необязательные поля:
\begin{itemize}
\item Все поля, описывающие отряд. Ценность и награды будут равномерно распределены по всем отрядам в группе. См. \hyperref[group]{\selectlanguage{Russian}Группа / Отряд}
\item \texttt{count} - количество случайных отрядов в группе
\item \texttt{owner} - раса, контролирующая отряды. По умолчанию отряды принадлежат нейтральной расе. См. \hyperref[raceTypes]{\selectlanguage{Russian}Расы}
\item \texttt{aiPriority} - приоритет отрядов для ИИ в диапазоне \texttt{[0 : 6]}
\item \texttt{name} - имя для лидеров отрядов. По умолчанию будет использовано имя из \texttt{TLeader.dbf}
\item \texttt{order} - приказ для отряда. По умолчанию \texttt{Order.Stand}. См. \hyperref[orderTypes]{\selectlanguage{Russian}Типы приказов}
\item \texttt{leaderIds} - список лидеров, один из которых будет выбран в качестве лидера отряда.
\item \texttt{leaderModifiers} - модификаторов для лидера.
\end{itemize}

Примеры:\\
12 случайных отрядов общей ценностью 1200.
Награда из зелий лечения и буста общей ценностью 200 - 300.
На 12 отрядов обязательно будут созданы  1 - 5 эликсиров восстановления.

\begin{figure}[H]
\lstinputlisting{docExamples/neutralStacksExample1.lua}
\end{figure}

5 случайных отрядов из воинов империи и нейтральных людей общей ценностью 1500 - 1600.

\begin{figure}[H]
\lstinputlisting{docExamples/neutralStacksExample2.lua}
\end{figure}