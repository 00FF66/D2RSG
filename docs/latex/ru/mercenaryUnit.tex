\subsubsection{Наемник}
\label{mercenaryUnit}
Описывает обязательного наемника в лагере.

Обязательные поля:
\begin{itemize}
\item \texttt{id} - идентификатор юнита из GUnits.dbf
\item \texttt{level} - уровень юнита. \textbf{\selectlanguage{Russian}Важно:} уровень юнита не может быть ниже базового!
\end{itemize}

Необязательные поля:
\begin{itemize}
\item \texttt{unique} - определяет будет ли юнит доступен для найма единожды
\end{itemize}

Пример:\\
'Рейнджер' (т2 лучник империи), доступный для найма бесконечное число раз

\begin{lstlisting}
{ id = 'g000uu0007', level = 2, unique = false }
\end{lstlisting}

'Скелет-воин' (т3 воин нежити) 20 уровня, доступный для найма единожды

\begin{lstlisting}
{ id = 'g000uu0088', level = 20, unique = true }
\end{lstlisting}