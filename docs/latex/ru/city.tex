\subsubsection{Город}
\label{city}
Описывает нейтральный город, его гарнизон и посещающий отряд, а также предметы внутри города.

Обязательные поля:
\begin{itemize}
\item \texttt{tier} - уровень города в диапазоне \texttt{[1: 5]}
\end{itemize}

Необязательные поля:
\begin{itemize}
\item \texttt{garrison} - защитники в гарнизоне города. См. \hyperref[group]{\selectlanguage{Russian}Группа / Отряд}. \textbf{\selectlanguage{Russian}Важно:} Предметы указанные для защитников будут созданы в инвентаре города.
\item \texttt{stack} - отряд посещающий город. См. \hyperref[group]{\selectlanguage{Russian}Группа / Отряд}
\end{itemize}

Пример:\\
Город 1 уровня с гарнизоном из гномов или нейтральных людей.
В гарнизоне будут созданы зелья лечения и бафов на 200 - 300 ценности.
Посещающий отряд в воротах города состоит из варваров и волков.

\begin{figure}[H]
\lstinputlisting{docExamples/cityExample.lua}
\end{figure}