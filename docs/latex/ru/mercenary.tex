\subsubsection{Лагерь наемников}
\label{mercenary}
Описывает лагерь наемников, юнитов в нем, а также потенциальную охрану. Юниты делятся на случайные и обязательные.

Обязательные поля:
\begin{itemize}
\item \texttt{value} - общая ценность случайных юнитов. См. \hyperref[value]{\selectlanguage{Russian}Ценность}
\end{itemize}

Необязательные поля:
\begin{itemize}
\item \texttt{subraceTypes} - список сабрас определяющий какие типы юнитов могут быть созданы. См. \hyperref[subraceTypes]{\selectlanguage{Russian}Субрасы}. В случае пустого списка могут быть созданы юниты любых субрас
\item \texttt{units} - список наемников которые обязательно должны быть в лагере. Не зависит от ценности случайных юнитов. См. \hyperref[mercenaryUnit]{\selectlanguage{Russian}Наемник}
\item \texttt{guard} - отряд охраняющий вход в лагерь. См. \hyperref[group]{\selectlanguage{Russian}Группа / Отряд}
\end{itemize}

Пример:\\
Лагерь наемников с юнитами империи и нейтральными людьми общей ценностью от 200 до 250.
В лагере обязательно будут доступны 'Рейнджер' и 'Скелет-воин' 20 уровня, доступный для найма лишь единожды

\begin{figure}[H]
\lstinputlisting{docExamples/mercenaryExample.lua}
\end{figure}