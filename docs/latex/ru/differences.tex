\section{Отличия от версии 0.7.0}
\begin{itemize}
\item в параметры шаблона добавлено поле \hyperref[customParameters]{{CustomParameters}} для описания дополнительных параметров шаблона, доступных для выбора пользователем.
\item добавлена поддержка дополнительных \hyperref[subraceTypes]{субрас}
\item в таблице \hyperref[mercenary]{\texttt{Лагерь наемников}} изменено поле \texttt{value} - теперь обозначает максимальную стоимость всех юнитов в лагере наемников, вместо опыта за убийство.
\item в таблицу \hyperref[mercenary]{\texttt{Лагерь наемников}} добавлено поле \texttt{enrollValue} для задания минимальной и максимальной цены юнита.
\item в таблицу \hyperref[getContents]{\texttt{getContents}} добавлено поле \texttt{scenarioVariables} для добавления переменных сценария.
\item в таблицу \hyperref[template]{\texttt{template}} добавлено поле \texttt{iterations} для изменения количества итераций по перемещению зон при генерации.
\item в таблицу \hyperref[neutralStacks]{\texttt{Нейтральный отряд}} добавлено поле \texttt{order} для задания приказа отряду.
\item в таблицу \hyperref[neutralStacks]{\texttt{Нейтральный отряд}} добавлено поле \texttt{leaderIds} для задания списка возможных лидеров отряда.
\item в таблицу \hyperref[neutralStacks]{\texttt{Нейтральный отряд}} добавлено поле \texttt{leaderModifiers} для задания списка модификаторов лидера.
\item в таблицы \hyperref[capital]{\texttt{Столица}}/\hyperref[city]{\texttt{Город}} добавлено поле \texttt{gapMask} для создания отступа от других объектов и блокирования создания гор с определенных сторон.
\item в таблицу \hyperref[capital]{\texttt{Столица}} добавлено поле \texttt{guardian} для определения будет ли сгенерирован страж столицы.
\item в таблицу \hyperref[connection]{\texttt{Соединения}} добавлено поле \texttt{size} для задания размера прохода(0 или 1).
\end{itemize}